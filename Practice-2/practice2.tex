\documentclass[12pt, a4paper]{article}
\usepackage[utf8x]{inputenc}
\usepackage{amsmath}
\usepackage[russian]{babel}
\usepackage[width=18cm, top=2cm, bottom=2cm]{geometry}
\usepackage{indentfirst}
\usepackage{enumerate}
\usepackage{amsfonts}
\usepackage{amssymb}
\usepackage{graphicx}
\usepackage{multirow}
\usepackage{pgf}
\usepackage{pgfplotstable}


\title{Задание №2\\
		Интерполяция функций полиномами}
\author{Анпилогов Асканий, 21.С03-мм}
%\date{1 марта 2023 г.}

\begin{document}
	\maketitle
	%\tableofcontents

\section*{Цель}
	\begin{itemize}
		\item Исследовать работу программы на функциях:
			\begin{enumerate}
				\item $f(x) = \frac{1}{1+x^2}$
				\item $f(x) = C_1, \, x_i = C_2$
			\end{enumerate}
		\item Выяснить, что будет происходить при росте числа интервалов интерполяции в случае использования равномерных и чебышевских сеток.
	\end{itemize}

\section*{Описание работы}

	\begin{tikzpicture}[scale=2]
		\begin{axis}[grid=major,legend pos=north west,
			xlabel={x},ylabel={$f(x)$}]
			\addplot table{v1/res_uniform.dat};
		\end{axis}
	\end{tikzpicture}

\end{document}

\documentclass[12pt, a4paper]{article}
\usepackage[utf8x]{inputenc}
\usepackage{amsmath}
\usepackage[russian]{babel}
\usepackage[width=18cm, top=2cm, bottom=2cm]{geometry}
\usepackage{indentfirst}
\usepackage{enumerate}
\usepackage{amsfonts}
\usepackage{amssymb}
\usepackage{graphicx}
\usepackage{multirow}
\usepackage{pgf}
\usepackage{pgfplotstable}


\title{Задание №2\\
		Интерполяция функций полиномами}
\author{Анпилогов Асканий,  21.С03-мм}
%\date{1 марта 2023 г.}

\begin{document}
	\maketitle
	\tableofcontents

	\section{Цель}
	\begin{itemize}
		\item Исследовать работу программы на функциях:
			\begin{enumerate}
				\item $f(x) = \frac{1}{1+x^2}$
				\item $f(x) = C_1, \, x_i = C_2$
			\end{enumerate}
		\item Выяснить, что будет происходить при росте числа интервалов интерполяции в случае использования равномерных и чебышевских сеток.
	\end{itemize}

	\section{Графики}

	\subsection{Тестовая функция}
	\begin{tikzpicture}
		\begin{axis}[
				width=16cm, height=7cm,
				grid=major, legend pos=south east,
				xlabel={$x$}, ylabel={$f(x)$}
			]
			\addplot table{res_test.dat};
			%\legend{$x = \overline{-6, 6}, \, y = 7, 3, 12, 4, 9, 1$};
		\end{axis}
	\end{tikzpicture}

	\subsection{$f(x) = \frac{1}{1+x^2}$}
	\begin{tikzpicture}
		\begin{axis}[
				width=16cm, height=7cm,
				grid=major, legend pos=south east,
				xlabel={$x$}, ylabel={$f(x)$}
			]
			\addplot table{res_uniform1.dat};
			\addplot[color=red,domain=-6:6,samples=100]{1/(1+x*x)};
			%\legend{$x = \overline{-6, 6}, \, y = 7, 3, 12, 4, 9, 1$};
		\end{axis}
	\end{tikzpicture}

	\subsection{$f(x) = C_1, \, x_i = C_2$}
	\begin{tikzpicture}
		\begin{axis}[
				width=16cm, height=7cm,
				grid=major, legend pos=south east,
				xlabel={$x$}, ylabel={$f(x)$}
			]
			\addplot table{res_uniform2.dat};
			\addplot[color=red,domain=-10:10,samples=100]{1};
			%\legend{$x = \overline{-6, 6}, \, y = 7, 3, 12, 4, 9, 1$};
		\end{axis}
	\end{tikzpicture}

	\section{Исправление ошибок}

	\subsection{uniform}
	\begin{tikzpicture}
		\begin{axis}[
				width=16cm, height=9cm,
				grid=major, legend pos=south east,
				xlabel={$x$}, ylabel={$f(x)$}
			]
			\addplot[color=blue] table{v1/res_uniform.dat};
			\addplot[color=red,domain=-10:10,samples=100]{1};
			%\legend{$x = \overline{-6, 6}, \, y = 7, 3, 12, 4, 9, 1$};
		\end{axis}
	\end{tikzpicture}

	\subsection{chebyshev}
	\begin{tikzpicture}
		\begin{axis}[
				width=16cm, height=9cm,
				grid=major, legend pos=south east,
				xlabel={$x$}, ylabel={$f(x)$}
			]
			\addplot[color=blue] table{v1/res_chebyshev.dat};
			\addplot[color=red,domain=-10:10,samples=100]{1};
			%\legend{$x = \overline{-6, 6}, \, y = 7, 3, 12, 4, 9, 1$};
		\end{axis}
	\end{tikzpicture}

\end{document}
